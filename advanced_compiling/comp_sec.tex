% !TeX program = xelatex

% \documentclass[]{article}


\documentclass[landscape]{article}
\usepackage{scrextend}
\KOMAoption{fontsize}{7pt}

\usepackage{enumitem}
\setlist[itemize,1]{leftmargin=0mm}

% \usepackage[T1]{fontenc}
% \usepackage{lmodern}
\usepackage{amssymb,amsmath}
\usepackage{ifxetex,ifluatex}
\usepackage{fixltx2e} % provides \textsubscript
% use upquote if available, for straight quotes in verbatim environments
% \IfFileExists{upquote.sty}{\usepackage{upquote}}{}
% \ifnum 0\ifxetex 1\fi\ifluatex 1\fi=0 % if pdftex
%   \usepackage[utf8]{inputenc}
% % % \else % if luatex or xelatex
% %   \ifxetex
% %     \usepackage{mathspec}
% %     \usepackage{xltxtra,xunicode}
% %   \else
% %     \usepackage{fontspec}
% %   \fi
% %   \defaultfontfeatures{Mapping=tex-text,Scale=MatchLowercase}
% %   \newcommand{\euro}{€}
% % % % % \fi
% use microtype if available
\IfFileExists{microtype.sty}{\usepackage{microtype}}{}
\ifxetex
  \usepackage[setpagesize=false, % page size defined by xetex
              unicode=false, % unicode breaks when used with xetex
              xetex]{hyperref}
\else
  \usepackage[unicode=true]{hyperref}
\fi
% \hypersetup{breaklinks=true,
%             bookmarks=true,
%             pdfauthor={Krishna Parashar},
%             pdftitle={Computer Security Field Guide},
%             colorlinks=true,
%             citecolor=blue,
%             urlcolor=blue,
%             linkcolor=magenta,
%             pdfborder={0 0 0}}
% \urlstyle{same}  % don't use monospace font for urls




\usepackage{multicol, calc, ifthen, amsmath, amsthm, amsfonts, amssymb, color, graphicx, fontspec, xunicode}

\usepackage[landscape]{geometry}

% \defaultfontfeatures{Mapping=tex-text,Scale=MatchLowercase}
\ifthenelse{\lengthtest { \paperwidth = 11in}}
    { \geometry{top=.3in,left=.3in,right=.3in,bottom=.3in} }
    {\ifthenelse{ \lengthtest{ \paperwidth = 297mm}}
        {\geometry{top=1cm,left=1cm,right=1cm,bottom=1cm} }
        {\geometry{top=1cm,left=1cm,right=1cm,bottom=1cm} }
    }
\pagestyle{empty}
\makeatletter
\setmainfont{Source Sans Pro}
\setmonofont{Menlo}
\DeclareMathSizes{3}{3}{2}{1}

\renewcommand{\section}{\@startsection{section}{1}{0mm}{-1ex plus -.5ex minus -.2ex}{0.5ex plus .2ex}{\normalfont\large\bfseries}}
\renewcommand{\subsection}{\@startsection{subsection}{2}{0mm}{-1explus -.5ex minus -.2ex}{0.5ex plus .2ex}{\normalfont\normalsize\bfseries}}
\renewcommand{\subsubsection}{\@startsection{subsubsection}{3}{0mm}{-1ex plus -.5ex minus -.2ex}{1ex plus .2ex}{\normalfont\small\bfseries}}
\makeatother
\setcounter{secnumdepth}{0}
\setlength{\parindent}{0pt}
\setlength{\parskip}{0pt plus 0.5ex}
\hypersetup{colorlinks=true, urlcolor=blue}
\def\ci{\perp\!\!\!\perp}






% \setlength{\parindent}{0pt}
% \setlength{\parskip}{6pt plus 2pt minus 1pt}
% \setlength{\emergencystretch}{3em}  % prevent overfull lines
\setcounter{secnumdepth}{0}

\title{Computer Security Field Guide}
\author{Krishna Parashar}
\date{\href{http://atrus.co}{Atrus}}

\begin{document}






\begin{multicols}{3}
\setlength{\premulticols}{1pt}
\setlength{\postmulticols}{1pt}
\setlength{\multicolsep}{1pt}
\setlength{\columnsep}{2pt}

\begin{center}
    \Large{\underline{Computer Security Field Guide}} \\
\end{center}
\begin{center}
    Written by: \href{http://krishna.im}{Krishna Parashar}\\
    Published by: \href{http://www.atrus.co}{Atrus}\\
\end{center}


% % \maketitle
% 
% \section{Notation}\label{notation}

All of the following uses the IA-32 (Intel 32-bit systems) Notation

Instructions are formatted at inst src dst

\begin{itemize}
\itemsep1pt\parskip0pt\parsep0pt
\item
  Registers are prefixed with \%
\item
  Constants are prefixed with \$
\item
  (\$exx) means accessing memory at register \%exx
\item
  l suffix for instructions that are 32-bit (long) instructions
\item
  SFP is the saved \%ebp on the stack
\item
  OFP is the old \%ebp from the previous stack frame
\item
  RIP is the return address on the stack
\end{itemize}

\subsection{Registers}\label{registers}

There are six general purpose registers: \%eax, \%ebx, \%ecx, \%edx,
\%edi, \%esi (\%eax stores return value)

\begin{enumerate}
\def\labelenumi{\arabic{enumi}.}
\itemsep1pt\parskip0pt\parsep0pt
\item
  \%ebp - base pointer, indicates start of stack frame (gen const for
  given function).
\item
  \%esp - stack pointer, indicates bottom on stack (can and does
  change).
\item
  \%eip - instruction pointer, indicates instruction to run.
\end{enumerate}

\subsection{Instructions}\label{instructions}

\begin{itemize}
\item
  mov a, b - copies value of a into b
\item
  push a - pushes a onto the stack (decrements stack, copies value over)
\item
  pop a - pop data from stack on a (copies value over and increments
  stack)
\item
  call func - pushes address of next instruction onto stack and
  transfers control to func
\item
  ret - pops return address of next instruction onto stack and transfers
  control to func
\item
  leave - mov \%ebp, \%esp then pop \%ebs (restores previous stack
  frame)
\item
  push1 \%ebp is part of the prologue for instructions that move the
  stack pointer to the current top of the stack. You then call movl
  \%esp, \%ebp to move \$ebp to where \%esp is. You do this at the
  beginning of each function call.
\item
  You push arguments in memory in reverse order.
\item
  Indexes in array are stored with the highest index immediately under
  the SFP, and lower values under that.
\end{itemize}

\section{Introduction}\label{introduction}

\section{Injection vulnerabilities, buffer overflows, and memory
safety}\label{injection-vulnerabilities-buffer-overflows-and-memory-safety}

\section{Software security}\label{software-security}

\section{Access control, OS security}\label{access-control-os-security}

\section{Privilege separation, security
principles}\label{privilege-separation-security-principles}

\section{Security principles}\label{security-principles}

\section{Web security: access control, same-origin
policy}\label{web-security-access-control-same-origin-policy}

\section{Web security: injection
vulnerabilities}\label{web-security-injection-vulnerabilities}

\section{Web security: XSS}\label{web-security-xss}

\section{Web security: session management and
CSRF}\label{web-security-session-management-and-csrf}

\section{Authentication and
impersonation}\label{authentication-and-impersonation}

\section{Web security: UI-based
attacks}\label{web-security-ui-based-attacks}

\section{Tracking on the web}\label{tracking-on-the-web}

\section{Symmetric-key encryption}\label{symmetric-key-encryption}

\section{Why This Exists}\label{why-this-exists}

While taking my Machine Structures class I found it very difficult to
conceptually understand and network the plethora of new found concepts.
Thus I wrote up this brief synopsis of the concepts I found useful to
understanding the core ideas. This is of course by no means
\textbf{comprehensive} but I do hope it will provide you with a somewhat
better understanding computer architecture. Please feel free to
\href{mailto:me@krishna.im}{email me} if you have an questions,
suggestions, or corrections. Thanks and enjoy!

\begin{center}\rule{3in}{0.4pt}\end{center}

\subsection{Introduction}\label{introduction-1}

Okay, so you want to \emph{understand} Machine Structures. But why in
heaven's name to you want to take on this rather insurmountable task?
I'll take a wild guess you may be forced into this by your universities'
curriculum task force-namely your professor. Despite the pain in
frustration you \emph{may} go through as you dive deeper, believe or not
the ideas in this realm are actually quite useful in your everyday life.
In fact the advances we have made in machine structures in the past
thirty years are the reason the internet exists in the capacity we have
grown to love. Because of this progress you can use things like
\emph{parallelism} and \emph{pipelining} run an intensive Google search
in milliseconds or execute massive projects like mapping the Human
Genome to tailor medical care specifically to you.

\begin{center}\rule{3in}{0.4pt}\end{center}

\subsection{The Big Picture}\label{the-big-picture}

So chances are you have already tried a bit of coding. But how does that
virtual code turn in to physical phenomenon? Well, let's start of by
defining a few ideas in the computing lexicon:

\begin{itemize}
\item
  An \textbf{Operating System (OS)} is a interface between a your
  program and the hardware that manages the resources and ensures you
  can do things like use a keyboard, store data in memory, and handle
  many applications at once.
\item
  A \textbf{Scripting Language} is probably what you did or want to
  learn first. Python or Ruby or Java are pretty fun examples. These
  scripting languages are named so because they try to look like you are
  writing an essay (that actually \emph{does} cool things) in a pretty
  logical and shorthand script. Want to print ``hello'' in Python? Here
  it is: \texttt{print ("hello")}! (Beautiful isn't it?)
\item
  A \textbf{High Level Language} is something you may or may not have
  written before. Lisp, C, C++ are all higher level languages. They may
  not be as beautiful as the Python code, but boy oh boy can you make
  the program run really, really fast. That same hello statement from
  Python? Well in C it looks like:
\end{itemize}

\begin{quote}
\texttt{char string{[}{]} = "Hello World";}
\end{quote}

\begin{quote}
\texttt{printf("\%s \textbackslash{}n", string);}
\end{quote}

\begin{quote}
(Now it's not as fun as before.)
\end{quote}

\begin{itemize}
\item
  A \textbf{Compiler} is a program that parses (goes through) your
  complex C or C++ code and turns it into something that's harder for
  you to read, but easier for a computer to read. FYI though, a
  \textbf{compiler} is an umbrella term that can also mean turning that
  beautiful Python code to a more complicated C version, but more
  generally used to mean from C to an Assembly Language.
\item
  An \textbf{Assembly Language} is the output from the compiler and
  looks like a funky short fragments such as:
\end{itemize}

\begin{quote}
\texttt{multi \$t2 \$t1 4}
\end{quote}

\begin{quote}
Don't worry if you don't understand what the above means. It is written
in a language called \textbf{MIPS} that we will discuss later. It is
worth mentioning that Intel has a very popular assembly language called
\textbf{x86}, which can get quite complex and unfortunately will not be
discussed in much detail in this guide.
\end{quote}

\begin{itemize}
\item
  An \textbf{Instruction} is each one of those funky little fragments
  from the compiler.
\item
  An \textbf{Assembler} is yet another program that takes in the
  assembly language and interprets that into something the \emph{CPU}
  can read and execute.
\item
  The \textbf{Machine Language} is the output from the assembler looks
  quite intimidating. Here is an example of what adding two things looks
  like:
\end{itemize}

\begin{quote}
\texttt{1000110010100000}
\end{quote}

\begin{quote}
Yep! You guessed it. It's binary! The CPU's structure (discussed lated)
need the format to be in just 1's and 0's for reasons in the realm of
mathematics and logic. Feel free to look it up, there is a lot of cool
information about that.
\end{quote}

\begin{itemize}
\item
  A \textbf{Binary Digit (Bit)} is well, the each one of those 1's and
  0's. Each spot where you can have a digit contains a \emph{bit} so if
  you have the above machine language output, that would be 16
  \textbf{bits}. Now you may be wondering, ``Hey I have heard of a
  \textbf{Byte}, is that the same as a \emph{bit}?'' Good question!
\item
  A \textbf{Byte} the what you call when you have a collection of 8
  \emph{bits}. So if you have the 16 bits of machine language we were
  talking about earlier, you can alternately say you have 2
  \textbf{bytes} of machine language. Pretty cool huh? That 500 GB hard
  drive? 500 * 1,000,000,000 (from the \emph{Giga} part) \textbf{bytes}
  or 500 * 1,000,000,000 * 8 (from the byte part) \textbf{bits}!
\item
  A \textbf{Transistor} is the physical representation of the 0 or 1
  bit. It is a physical digital \emph{switch}. When it is flipped
  \emph{ON} (current going through it) it used to mean a \emph{1}, and
  when it is \emph{OFF} (no current) it represents a \emph{0}.
\end{itemize}

Phew! Now that that's out of the way, we can starting talking about some
really useful and brilliant uses of these things. Here is a quick visual
summary:

\begin{center}\rule{3in}{0.4pt}\end{center}

\subsection{The Six Great Ideas in Computer
Architecture}\label{the-six-great-ideas-in-computer-architecture}

It is now that we come upon the \emph{The Six Great Ideas in Computer
Architecture} (so named due to their greatness). These topics will form
the basis for the rest of this guide.

\begin{enumerate}
\def\labelenumi{\arabic{enumi}.}
\item
  \textbf{Design processors using
  \href{http://wikipedia.org/wiki/Moores_law}{Moore's Law}} which states
  processing speed and memory capacity will double every two years; an
  occurrence that is related to the number of transistors in the chip
  doubling. It was predicted by Gordon Moore in 1965, and has held
  roughly true thus far.
\item
  \textbf{Abstract as much as possible in order to simplify the design.}
  This idea relates to what we talked about in the previous section.
  Split up the responsibility of understanding your code by using a
  standardized hierarchy (High Level Language -\textgreater{} Assembly
  Language -\textgreater{} Machine Level) so that each layer takes in an
  input and passes it down to the next layer. This process makes it
  infinitely easier to find out where things went or can go wrong.
\item
  \textbf{Design so that the most common case is fast.} This one should
  be rather intuitive. Instead of wasting time, energy, and of course
  money trying to optimize so that every part of the program (down to
  the tiny edge cases) is blazing fast, why not just make the most used
  cases faster? Otherwise you'll end up with a lot of code that is
  probably only trivially faster than if you just made the most often
  used cases faster.
\item
  \textbf{Make dependable systems by using redundancy.} This one is also
  rather intuitive. Basically you should make backups of the data and
  repeat the work with other parts of the computer system to ensure
  everything is accurate and dependable (nothing fails).
\item
  \textbf{Use the capacities and speeds of different storage systems to
  make things fast.} This is actually one of the main ideas in this
  guide. We use this principal to optimize the usage of different kinds
  of memory (fast vs.~slow, big vs.~small) to cleverly and thriftily use
  the resources and make our programs fast and light.
\item
  \textbf{Find ways to improve performance} using techniques such as
  \textbf{Parallelism, Pipelining, and Prediction}. The techniques will
  also be discussed in greater detail later on. The basic idea for each
  is to have multiple parts of the computer to split up the work
  (\textbf{parallelism}), stage the processes so that no part of the
  computer has the excuse that it wasn't told what to do
  (\textbf{pipelining}), and lastly try to predict where along the
  pipeline things might fail and proactively prevent those failures
  (\textbf{prediction}).
\end{enumerate}

With basically these ideas we have managed to come to where computers
are today! Pretty impressive, isn't it?

\begin{center}\rule{3in}{0.4pt}\end{center}

\subsection{The Hardware Structure of A
Computer}\label{the-hardware-structure-of-a-computer}

\subsubsection{Introduction}\label{introduction-2}

Now we want to try and understand how a modern computer is structured
today. We will choose a relatively simple example, but don't fear if you
don't understand what something is. All will come in due time! Thus we
begin.

Talk about the components, CPU, SRAM, DRAM, Hard Disks, I/O

\begin{center}\rule{3in}{0.4pt}\end{center}

\subsection{Memory Hierarchy}\label{memory-hierarchy}

\subsubsection{Introduction}\label{introduction-3}

One of the most fascinating ideas in the structure of computers is
Memory. We know from the previous section that memory is the part of the
computer that allows data to be stored and accessed rather quickly. It
is a step down from Cache (in the CPU so super fast) and a step up from
Hard Disks (tons of Capacity but takes longer to access). Ideally we
would want an indefinitely large memory capacity such that any
particular data we want would would be immediately available. However in
order to reduce costs (memory is very expensive), we have developed a
bunch of tricks and techniques to try to optimize the situation and get
the most out of this bad bargain.

\subsubsection{Memory Hierarchy}\label{memory-hierarchy-1}

So now we come across our first trick \emph{Memory Hierarchy}. This
structure (depicted below) that uses multiple levels of memories with
different speeds and sizes efficiently in order to present the user with
as much memory as is available in the cheapest technology, while
providing access at the speed offered by the fastest memory.
Unfortunately as the distance from the processor increases, both the
size and access time of memories increase.

Data is similarly hierarchical; a level closer to the processor is
generally a subset of any level further away, and the entirety of the
data is stored at the lowest level (generally a hard drive)

\subsubsection{Principle of Locality}\label{principle-of-locality}

So in order to allow programs to run efficiently, a technique we take
advantage of is the \emph{Principles of Locality}. Programs accesses a
relatively portion of their address space at any instant of time. We
take advantage of the \emph{Memory Hierarchy} from above. There are two
different types of locality:

\begin{enumerate}
\def\labelenumi{\arabic{enumi}.}
\itemsep1pt\parskip0pt\parsep0pt
\item
  \textbf{Temporal Locality} (locality in time) \emph{is the ideas that
  if an item is referenced we will assume it will tend to be referenced
  again soon.} An Example of this is \emph{Loops} in which that
  variables and functions are likely to be accessed repeatedly so we
  save them closer to the processing.
\item
  \textbf{Spatial Locality} (locality in space) \emph{is the idea that
  if an item is referenced, items whose addresses are close by will
  probably be referenced sometime soon.} Again we can use loops as an
  example, since we know that instructions are normally accessed
  sequentially so we can predict the next iteration and store it when we
  access the previous iteration. We can also think of this using array
  elements. We can save the nearby array elements when we get one of the
  elements with hopes that it will come in handy. It may seem a bit
  counterintuitive but it really does work!
\end{enumerate}

In short by using these two ideas (\emph{Memory Hierarchy} and the
\emph{Principle of Locality}) uses smaller and faster memory
technologies close to the processor. Thus, accesses that hit in the
highest level of the hierarchy can be processed quickly. Accesses that
miss go to lower levels of the hierarchy which are larger but slower.

\subsubsection{Memory Access \textless{}- NEEDS
WORK}\label{memory-access---needs-work}

A memory hierarchy usually consist of many different levels as shown in
the \emph{Memory Hierarchy} section. However Data can only be copied
between only two adjacent levels at a time. Thus to show how this data
is accessed we will show how one particular instance between two levels
is done.

\textbf{Block (Line)}: The minimum unit of information that can be
either present or not present in cache. This is shown in the image above
as the red square.

Every pair of levels in the memory hierarchy can be thought of as having
an:

\begin{itemize}
\itemsep1pt\parskip0pt\parsep0pt
\item
  \textbf{Upper Level}:

  \begin{itemize}
  \itemsep1pt\parskip0pt\parsep0pt
  \item
    Closer to the processor
  \item
    Smaller, faster, and more expensive than the lower level
  \item
    Data in this level is generally a subset of the lower level
  \end{itemize}
\item
  \textbf{Lower Level}
\end{itemize}

\paragraph{SRAM Technology}\label{sram-technology}

\begin{itemize}
\itemsep1pt\parskip0pt\parsep0pt
\item
  Static Random Access Memory
\item
  Integrated circuits that form memory arrays
\item
  Usually has a single access port that can provide either a read or a
  write
\item
  Fixed access time to any datum (read/write access times may differ)
\item
  Typically uses 6-8 transistors per bit to prevent information
  disruption when read
\item
  As long as power is applied the value can be kept indefinitely
\end{itemize}

\paragraph{DRAM Technology}\label{dram-technology}

\begin{itemize}
\itemsep1pt\parskip0pt\parsep0pt
\item
  Dynamic Random Access Memory
\item
  Value keep in a cell is stored as a charge in a capacitor
\item
  Single transistor is used to access stored charge (either to read or
  overwrite the stored charge)

  \begin{itemize}
  \itemsep1pt\parskip0pt\parsep0pt
  \item
    Because it uses one transistor per bit of storage, it is much denser
    and cheaper than SRAM
  \end{itemize}
\item
  Stores charge on a capacitor: it cannot be kept indefinitely and must
  be periodically refreshed
\item
  \emph{Refresh}: contents from an entire row is read and immediately
  written back to the same row
\end{itemize}

\subsubsection{Performance Specifications \textless{}- Why? Seems
Unnecessary.}\label{performance-specifications---why-seems-unnecessary.}

\begin{itemize}
\itemsep1pt\parskip0pt\parsep0pt
\item
  Buffer Rows:

  \begin{itemize}
  \itemsep1pt\parskip0pt\parsep0pt
  \item
    Acts like a SRAM: changing the address enables random bits access to
    be accessed until the next row is accessed
  \end{itemize}
\item
  Wider Chips:

  \begin{itemize}
  \itemsep1pt\parskip0pt\parsep0pt
  \item
    Improves memory bandwidth
  \end{itemize}
\item
  Organization:

  \begin{itemize}
  \itemsep1pt\parskip0pt\parsep0pt
  \item
    Modern DRAMs are organized in banks
  \item
    \emph{Bank}: Series of rows

    \begin{itemize}
    \itemsep1pt\parskip0pt\parsep0pt
    \item
      Pre-charge opens/closes a bank
    \end{itemize}
  \item
    DDR3: 4 banks
  \item
    \emph{Act }: Signal sent with row addresses that activates the
    transfer of the row to the buffer
  \item
    When row is in the buffer, it can be transferred by

    \begin{itemize}
    \itemsep1pt\parskip0pt\parsep0pt
    \item
      Successive column addresses at whatever the width of the DRAM is
      (typically 4, 8 or 16)
    \item
      Specifying a block transfer and the starting address
    \end{itemize}
  \end{itemize}
\item
  Clocks Added:

  \begin{itemize}
  \itemsep1pt\parskip0pt\parsep0pt
  \item
    SDRAM (Synchronous DRAM)
  \item
    Eliminates synchronization time between memory and processor
  \item
    Speed Advantage: transfers bits in the burst without having to
    specify additional address bits
  \item
    DDR SDRAM (Double Data Rate):

    \begin{itemize}
    \itemsep1pt\parskip0pt\parsep0pt
    \item
      Data transfers on both the rising and falling edge of the clock
      (twice bandwidth)
    \item
      Latest version: DDR4 can do 3200 million transfers per second
      (1600 MHz clock)
    \end{itemize}
  \end{itemize}
\item
  Address Interleaving

  \begin{itemize}
  \itemsep1pt\parskip0pt\parsep0pt
  \item
    Instead of just a faster row buffer, DRAM can read from or write to
    multiple banks, each having its own row buffer
  \item
    Accesses rotation: Enables sending addresses to several banks to
    read/write simultaneously
  \item
    Bandwidth = Bandwidth x (\# of banks)
  \end{itemize}
\end{itemize}

\subsubsection{Flash Memory}\label{flash-memory}

\subsubsection{Disk Memory}\label{disk-memory}

\begin{itemize}
\itemsep1pt\parskip0pt\parsep0pt
\item
  Wear Leveling

  \begin{itemize}
  \itemsep1pt\parskip0pt\parsep0pt
  \item
    Like other EEPROM technologies, writes can wear out flash memory
    bits
  \item
    To handle this a controller is used to spread the writes by
    remapping blocks that have been written many times to less used
    blocks
  \item
    With this technology mobile devices are very unlikely to exceed
    flash's write limits
  \item
    To improve performance even more, incorrectly manufactured memory
    cells are mapped out
  \item
    Wear Leveling lowers flash's potential performance, but is needed
    unless higher-level software monitors block wear
  \end{itemize}
\item
  Magnetic Hard Disks consist of a collection of platters
\item
  Metal platters are covered with magnetic recording material on both
  sides
\item
  Track: One of thousands of concentric circles that makes up the
  surface of a magnetic disk
\item
  Sector:

  \begin{itemize}
  \itemsep1pt\parskip0pt\parsep0pt
  \item
    One of the segments that make up a track on a magnetic disk
  \item
    The smallest amount of information that is read/written on a disk
  \end{itemize}
\item
  Read/Write mechanism

  \begin{itemize}
  \itemsep1pt\parskip0pt\parsep0pt
  \item
    Read-Write Head: Movable arm containing a small electromagnetic coil
  \item
    Each surface has one arm containing two RW-Heads, one facing up and
    one facing down
  \end{itemize}
\end{itemize}

\subsubsection{Virtual Memory}\label{virtual-memory}

\begin{center}\rule{3in}{0.4pt}\end{center}

\subsection{Caches}\label{caches}

\subsubsection{Structure}\label{structure}

Registers, Etc.

\subsubsection{Hitting and Missing \textless{}- NEEDS
WORK}\label{hitting-and-missing---needs-work}

We know from the \emph{The Hardware Structure of A Computer} section
that caches live in the CPU and live near the top of Memory Hierarchy
pyramid (registers are faster). So before we move on and discuss caches,
we want to clear up some important terminology.

\begin{itemize}
\itemsep1pt\parskip0pt\parsep0pt
\item
  \emph{Hit}: When data requested by the processors is found in some
  block in the upper level
\item
  \emph{Hit Rate}: Fraction of memory accesses found in a level of the
  memory hierarchy
\item
  \emph{Hit Time}: Time required to access a level of the memory
  hierarchy, including the time needed to determine whether the access
  time is a hit or a miss
\item
  \emph{Miss}: When data requested by the processor is not found in the
  upper level -\textgreater{} lower level accessed to retrieve the block
  containing the requested data
\item
  \emph{Miss Rate}: Fraction of memory accesses not found in a level of
  the memory hierarchy (1 - Hit Rate)
\item
  \emph{Miss Penalty}: Time required to fetch a block from a lower level
  of the memory hierarchy, including --- time to access the block ---
  transmit it from one level to the other --- insert it in the level
  that experienced the miss --- pass the block to the requestor \#\#\#
  AMAT
\end{itemize}

\subsubsection{Direct Mapped Caches}\label{direct-mapped-caches}

\subsubsection{Multilevel Caches}\label{multilevel-caches}

\subsubsection{Set Associative Caches}\label{set-associative-caches}

\begin{center}\rule{3in}{0.4pt}\end{center}

\subsection{Machine Instructions}\label{machine-instructions}

\subsubsection{Introduction}\label{introduction-4}

\subsubsection{MIPS}\label{mips}

Basics MIPS stuff e.g.~how it works, instructions, link to green sheet,
and uses Converting Binary to MIPS, Parity Bits, and those table things
and everything. CS61C Winston's Discussion will be helpful.

\begin{center}\rule{3in}{0.4pt}\end{center}

\subsection{Binary Representations}\label{binary-representations}

\subsubsection{Introduction}\label{introduction-5}

Signed, Unsigned, Two's Complement, Hex, Conversions, Extensions etc.

\begin{center}\rule{3in}{0.4pt}\end{center}

\subsection{Hardware Level}\label{hardware-level}

\subsubsection{Introduction}\label{introduction-6}

\subsubsection{Transistors and Logic
Gates}\label{transistors-and-logic-gates}

\subsubsection{Boolean Algebra}\label{boolean-algebra}

\subsubsection{}\label{section}

Assembly to Machine, Logic Gates,
\href{https://www.cise.ufl.edu/~mssz/CompOrg/CDA-proc.html}{Useful
Processor Stuff}

\begin{center}\rule{3in}{0.4pt}\end{center}

\subsection{The CPU}\label{the-cpu}

\subsubsection{Structure}\label{structure-1}

Purpose? Multiprocessing ALU, shifters, Reg files, Muxing, demuxming,
FSM \#\#\# Performance CPU Time, CPI, Instruction Count, Clock Cycle
time?

\begin{center}\rule{3in}{0.4pt}\end{center}

\subsection{Code Optimization
Techniques}\label{code-optimization-techniques}

\subsubsection{Introduction}\label{introduction-7}

\subsubsection{Cache Blocking}\label{cache-blocking}

\subsubsection{Pipelining}\label{pipelining}

Lead to next section : Parallelism

\begin{center}\rule{3in}{0.4pt}\end{center}

\subsection{Parallelism}\label{parallelism}

\subsubsection{Introduction}\label{introduction-8}

\subsubsection{Amdahl's Law}\label{amdahls-law}

\subsubsection{Request Level
Parallelism}\label{request-level-parallelism}

\paragraph{Application: MapReduce}\label{application-mapreduce}

\subsubsection{Data Level Parallelism}\label{data-level-parallelism}

\subsubsection{Flynn Taxonomy}\label{flynn-taxonomy}

\subsubsection{Thread Level Parallelism}\label{thread-level-parallelism}

\subsubsection{Shared Memory}\label{shared-memory}

\subsubsection{Application: OpenMP (?)}\label{application-openmp}

\subsubsection{Instruction Level
Parallelism}\label{instruction-level-parallelism}

\subsubsection{Application: Warehouse Scale
Computing}\label{application-warehouse-scale-computing}

\begin{center}\rule{3in}{0.4pt}\end{center}

\subsection{Redundancy}\label{redundancy}

ECC, RAID, Stuff from Professor Katz's lecture.

\begin{center}\rule{3in}{0.4pt}\end{center}

\subsection{Virtual Machines (?)}\label{virtual-machines}

\begin{center}\rule{3in}{0.4pt}\end{center}

\subsection{Colophon}\label{colophon}

Written by \href{http://krishna.im}{Krishna Parashar} in Markdown on
Byword. Used \href{http://johnmacfarlane.net/pandoc/}{Pandoc} to convert
from Markdown to Latex.


% % 





\end{multicols}
\end{document}